\chapter{Charge Pumps}
Charge pumps are used for power conversion, similar to buck and boost converters. The difference between converters and charge pumps is that the latter store energy using capacitors while the former store energy in inductors. This means charge pumps are generally much smaller (inductors usually dominate the size of converters) and can be integrated. Converters have good performance over a wide range of input voltages while charge pumps are limited to a much narrower range. Consequently, converters are usually used for discrete circuits while charge pumps are typically integrated. On the other hand, they are typically simpler than converters and so they may be more suitable for simpler discete designs.
\par
Like a boost converter, a charge pump can be used to generate a voltage that is higher than its input voltage (often the supply voltage). It can also be used to generate a negative voltage. Since power must be conserved, of course, the output current is always less than the input current. Charge pumps are typically used in ICs to generate voltages above the positive supply voltage or below the negative supply voltage (or $GND$). Charge pumps are also useful for generating programming voltages for non-volatile memories (such as EEPROM).

%\section{Simple voltage doubler}

%\section{Simple voltage doubler with transistors}

%\section{Simple voltage tripler}

%\section{Simple voltage inverter}

%\section{Voltage doubler with transistors}