\chapter{Filters}
This chapter presents a variety of filters using either passive or active components.
First and second order filters are presented so that an $n$th-order filter can be constructed by cascading these first and/or second order filters.

Some of these filters can be designed either as Butterworth or Chebyshev filters.
Butterworth filters are characterized by their maximally flat magnitude in the pass band for all-pole filters (so that, for all-pole filters, they are the best approximation to an ideal filter in the pass band).
Unfortunately, the magnitude of a Butterworth filter poorly approximates an ideal filter near the cutoff frequency in that the magnitude does not drop particularly sharply from the pass band to the stop band.
Higher order Butterworth filters transition more sharply and thus better approximate an ideal filter, but they are still inferior at the cutoff frequency to filters like the Chebyshev filter.
Chebyshev filters' magnitude response best approximates an ideal filter for all-pole filters in that the magnitude drops very sharply from the pass band to the stop band, but their frequency response has ripples in the pass band (i.e. the magnitude oscillates in the pass band, particularly near the cutoff frequency).
Higher order Chebyshev filters transition even more sharply than lower order Chebyshev filters.
Thus, Butterworth and Chebyshev filters solve different problems -- the former approximates an ideal filter best in the pass band while the latter is a better approximation near the cutoff frequency. \autocite[107, 111]{op-amp-circuits-johnson}
% figure out where to put the tables for the coefficients

Filter transfer functions are given as a function of the complex frequency-domain variable \(s = \sigma + \jomega\), where the angular frequency \(\omega\) in \si{\radian/\second} is related to the frequency \(f\) in \si{\Hz} as

\begin{equation}
	f = \frac{\omega}{2\pi}
	\label{eq:omega_to_f}
\end{equation}

The filter transfer function, typically denoted \Hs, can be expressed as a rational function of polynomials with real coefficients:

\begin{equation}
	\Hs = \Hpass\frac{\sum\limits_{i=0}^{m} a_i s^i}{\sum\limits_{i=0}^n b_i s^i}
	\label{eq:general_filter_TF}
\end{equation}

where \Hpass is the passband gain \autocite[8.5]{basic-linear-design-adi}.
The degree of the denominator determines the order of the filter, and a higher order is required to achieve a sharper cutoff from passband to stopband.

A passive filter (which uses no active components) is limited to \(\abs{\Hpass} \leq 1\).
Additionally, passive filter sections often require (active) buffer(s) at their input and/or output to isolate them from a high source impedance and/or low load impedance, respectively.
Furthermore, higher order filters are often constructed by cascading multiple lower order filters -- and buffers will likely need to be inserted between lower order filters if they are passive.
The result is that all but the simplest filters require active component(s).
Analyses of passive filters in this chapter assume low source and high load impedances.

Filters' transfer functions are characterized by the cutoff frequency \fcutoff, angular cutoff frequency \wcutoff, and/or the time constant \(\tau\) in \si{\second}.
\fcutoff and \wcutoff are related by \eqref{eq:omega_to_f}, and \(\tau\) to the cutoff frequencies by

\begin{equation}
	\tau = \frac{1}{\wcutoff} = \frac{1}{2\pi\fcutoff}
	\label{eq:tau_to_cutoff_frequency}
\end{equation}

Second and higher order filters are also characterized by the dimensionless ``quality factor'' \Q or alternatively as the damping ratio

\begin{equation}
	\alpha = \frac{1}{Q}
	\label{eq:filter_Q_to_damping_ratio1}
\end{equation}

The damping ratio may alternatively be expressed as

\begin{equation}
	\zeta = 2\alpha = \frac{2}{Q}
	\label{eq:filter_Q_to_damping_ratio2}
\end{equation}

A filter with high \Q is underdamped, exhibiting peaking in the frequency response and oscillation that eventually converges to a steady-state value in the step response \autocite[8.7]{basic-linear-design-adi}.

Below are general forms for filters of various types and orders \autocite[for second-order equations see][8.13]{basic-linear-design-adi}:

% TODO: improve these by putting them in a table of numbered equations somehow?
% TODO: add alternative versions using \alpha and/or \zeta?
\begin{equation}
	\text{1st-order \acs{lpf}: }\Hs = \Hpass\frac{\wcutoff}{\wcutoff + s} = \Hpass\frac{1}{1 + \tau s}
	\label{eq:1storderLPF}
\end{equation}

\begin{equation}
	\text{1st-order \acs{hpf}: }\Hs = \Hpass\frac{s}{\wcutoff + s} = \Hpass\frac{\tau s}{1 + \tau s}
	\label{eq:1storderHPF}
\end{equation}

\begin{equation}
	\text{2nd-order \acs{lpf}: }\Hs = \Hpass \frac{\wcutoff^2}{s^2 + \frac{\wcutoff}{Q}s + \wcutoff^2}
	\label{eq:2ndorderLPF}
\end{equation}

\begin{equation}
	\text{2nd-order \acs{hpf}: }\Hs = \Hpass \frac{s^2}{s^2 + \frac{\wcutoff}{Q}s + \wcutoff^2}
	\label{eq:2ndorderHPF}
\end{equation}

\begin{equation}
	\text{2nd-order \acs{bpf}: }\Hs = \Hpass \frac{\frac{\wcutoff}{Q}s}{s^2 + \frac{\wcutoff}{Q}s + \wcutoff^2}
	\label{eq:2ndorderBPF}
\end{equation}

\begin{equation}
	\text{2nd-order \acs{brf}: }\Hs = \Hpass \frac{s^2 + \w[z]^2}{s^2 + \frac{\wcutoff}{Q}s + \wcutoff^2}
	\label{eq:2ndorderBRF}
\end{equation}

% TODO modify the 2nd order circuits below to follow the symbols used in these general equations.

Since \s is proportional to frequency, we can determine a filter's low frequency response by evaluating its \Hs as \(s \to 0\) and its high frequency response by evaluating its \Hs as \(s \to \infty\).

\section{First-order low and high pass filters}
\subsection{First-order \(RC\) passive \acl{lpf}}
\begin{center}
	\begin{circuitikz}
		\draw (0,0) node[above]{\vin} to[R=$R$, o-*] ++(2,0) coordinate(join)
		to[C=$C$] ++(0,-2) node[ground]{}
		(join) to[short, *-o] ++(1,0) node[above]{\vout};
	\end{circuitikz}
\end{center}

Viewing the resistor and capacitor as an impedance divider:
\[\vout = \frac{\frac{1}{sC}}{R+\frac{1}{sC}}\vin\]
Rearranging, this is

\begin{equation}
	\frac{\vout}{\vin}(s) = \Hs = \frac{1}{1 + sRC}
	\label{eq:rcLPF}
\end{equation}

By inspection its cutoff frequency is \(\wcutoff = 1/(R C)\).

\subsection{First-order \(RC\) passive \acl{hpf}}
\begin{center}
	\begin{circuitikz}
		\draw (0,0) node[above]{\vin} to[C=$C$, o-*] ++(2,0) coordinate(join)
		to[R=$R$] ++(0,-2) node[ground]{}
		(join) to[short, *-o] ++(1,0) node[above]{\vout};
	\end{circuitikz}
\end{center}

Compared to the \ac{lpf}, the resistor and capacitor have been switched so the impedance divider is
\[\vout = \frac{R}{R+\frac{1}{sC}}\vin\]
Rearranged, this is

\begin{equation}
	\frac{\vout}{\vin}(s) = \Hs = \frac{sRC}{1 + sRC}
	\label{eq:rcHPF}
\end{equation}

By inspection its cutoff frequency is \(\wcutoff = 1/(R C)\).

\subsection{First-order non-inverting active \acl{lpf}}
\begin{center}
	\begin{circuitikz}
		\draw (0,0) node[above]{\vin} to[R=$R_1$, o-*] ++(2,0) coordinate(in)
		to[C=$C$] ++(0,-2) node[ground]{}
		(in) to[short] ++(2,0) node[op amp, noinv input up, anchor=+](OA){}
		(OA.out) to[R=$R_3$] ++(0,-2) coordinate(FB)
		to[R=$R_2$] ++(0,-2) node[ground]{}
		(OA.out) to[short, *-o] ++(1,0) node[above]{\vout}
		(OA.-) -- (FB -| OA.-) to[short, -*] (FB);
	\end{circuitikz}
\end{center}

This circuit is simply a cascade of the \(RC\) passive \ac{lpf} and the non-inverting amplifier.
The transfer function is thus

\begin{equation}
	\frac{\vout}{\vin}(s) = \Hs = \left(1+\frac{R_3}{R_2}\right)\left(\frac{1}{1 + sR_1 C}\right)
	\label{eq:1storderLPfilter}
\end{equation}

As usual, the resistors should be chosen such that \(R_1 = R_2 \parallel R_3\) to minimize the error due to the op amp's input bias currents.

In this circuit \R[1] and \C perform the same filtering function as the passive \(RC\) filter so the op amp isn't strictly necessary -- however, the op amp provides gain for the filter and can provide isolation of the \(RC\) filter from other circuit elements which might affect its performance.
If gain is not needed, the op amp can be configured as a voltage buffer by removing \R[2] and replacing \R[3] with a short.

% TODO: account for nonzero Rs? Then it should be Rs + R1 = R2 || R3
If \R[2] and \R[3] are used, the three resistors should be chosen such that the filter formed by \R[1] and \C has the desired cutoff frequency and \(R_1 = R_2 \parallel R_3\) (the latter relationship minimizes the error due to the op amp's input bias currents).

\subsection{First-order non-inverting active \acl{hpf}}
\label{sec:active1storderHPF}
\begin{center}
	\begin{circuitikz}
		\draw (0,0) node[above]{\vin} to[C=$C$, o-*] ++(2,0) coordinate(in)
		to[R=$R_1$] ++(0,-2) node[ground]{}
		(in) to[short] ++(2,0) node[op amp, noinv input up, anchor=+](OA){}
		(OA.out) to[R=$R_3$] ++(0,-2) coordinate(FB)
		to[R=$R_2$] ++(0,-2) node[ground]{}
		(OA.out) to[short, *-o] ++(1,0) node[above]{\vout}
		(OA.-) -- (FB -| OA.-) to[short, -*] (FB);
	\end{circuitikz}
\end{center}

This circuit is simply a cascade of the \(RC\) passive \ac{hpf} and the non-inverting amplifier.
The transfer function is thus

\begin{equation}
	\frac{\vout}{\vin}(s) = \Hs = \left(1 + \frac{R_3}{R_2}\right)\left(\frac{sR_1 C}{1 + sR_1 C}\right)
	\label{eq:active1storderHPF}
\end{equation}

As usual, the resistors should be chosen such that \(R_1 = R_2 \parallel R_3\) to minimize the error due to the op amp's input bias currents.

In this circuit \R[1] and \C perform the same filtering function as the passive \(RC\) filter so the op amp isn't strictly necessary -- however, the op amp provides gain for the filter and can provide isolation of the \(RC\) filter from other circuit elements which might affect its performance.
If gain is not needed, the op amp can be configured as a voltage buffer by removing \R[2] and replacing \R[3] with a short.

\subsection{First-order inverting active \acl{lpf}}
\label{sec:inverting_active_lpf_1st}
\begin{center}
	\begin{circuitikz}
		\draw (0, 0) node[op amp] (opamp) {}
		(opamp.-) to[short,-*] ++(-0.5,0) coordinate(VG)
		(opamp.+) to[short] ++(-0.5,0) (opamp.+ -| VG) to[R,l_=$R\sub{A}$] ++(0,-2) node[ground](G){}
		(opamp.out) to[short,*-o] ++(1,0)
		(VG) to[R,l_=$R$] ++(-2,0) to[short,-o] ++(-0.5,0) node[above]{\vin}
		(VG) to[short,-*] ++(0,1.25) coordinate(VGU) |- ++(0.5,0) coordinate(FL) to[R=$R\sub{F}$] ++(2,0) -| (opamp.out) to[short] ++(1,0) node[above]{\vout}
		(VGU) to[short] ++(0,1.5) coordinate(UFL) to[short] (FL |- UFL) to[C=$C$] ++(2,0) to[short] (UFL -| opamp.out) to[short,-*]  (VGU -| opamp.out);
	\end{circuitikz}
\end{center}

This circuit is the same as the inverting amplifier from \autoref{sec:inverting_amplifier}, but with a capacitor in parallel with the feedback resistor so that the feedback impedance is

\begin{equation*}
	\Z[F] = \R[F] \parallel C = \frac{\R[F]/(sC)}{\R[F] + 1/(sC)} = \frac{\R[F]}{1 + s\R[F]C}
\end{equation*}

Substituting this impedance into \eqref{eq:inverting_amplifier_TF} we obtain

\begin{equation}
	\frac{\vout}{\vin}(s) = \Hs = -\frac{\R[F]}{R}\left(\frac{1}{1 + s\R[F]C}\right)
	\label{eq:1storder_inverting_active_LPF}
\end{equation}

To minimize the error due to the op amp's input bias current, set \(\R[A] = R \parallel \R[F]\);
if this error is negligible \R[A] can be replaced with a short circuit.

% TODO: add from page 8.88 of the ADI filters chapter
% \subsection{First-order inverting active \acl{hpf}}

\section{Integrator and Differentiator}
\subsection{Inverting Integrator}
An ideal but impractical integrator circuit is:

\begin{center}
	\begin{circuitikz}
		\draw (0, 0) node[op amp] (opamp) {}
		(opamp.-) to[short,-*] ++(-0.5,0) coordinate(VG)
		(opamp.+) to[short] ++(-0.5,0) (opamp.+ -| VG) to[short] ++(0,-0.5) node[ground](G){}
		(opamp.out) to[short,*-o] ++(1,0)
		(VG) to[R,l_=$R$] ++(-2,0) to[short,-o] ++(-0.5,0) node[above]{\vin}
		(VG) to[short] ++(0,1.25) coordinate(VGU) |- ++(0.5,0) coordinate(FL) to[C=$C$] ++(2,0) -| (opamp.out) to[short] ++(1,0) node[above]{\vout};
	\end{circuitikz}
\end{center}

The inverting input's voltage is at \gnd and the non-inverting input is at a virtual ground.
The voltage across \R is thus equal to \(\vin(t)\) and the current through it is \[\iin(t) = \vin(t)/\R\]
By \ac{kcl} on the inverting input node and the fact that the input bias current into the op amp's inverting input is negliglible, \(\iin(t)\) is also the current through the capacitor.
The voltage across \C is \(-\vout(t)\) so \[\iin(t) = -C\odv{\vout(t)}{t}\]

Equating the two expressions for \(\iin(t)\) and solving for \(\vout(t)\):

\begin{equation}
	\vout(t) = \frac{-1}{RC}\int \vin(t) \odif{t} + \VOUT
	\label{eq:ideal_integrator_time}
\end{equation}

where \VOUT is the initial output voltage.

The transfer function in the frequency domain can be derived using \ac{kcl} at the inverting input node and the capacitor's impedance \(1/(sC)\):

\begin{equation}
	\frac{\vout}{\vin}(s) = \Hs = -\frac{1}{sRC}
	\label{eq:ideal_integrator_TF}
\end{equation}

Unfortunately, \C is an open circuit at \DC so the ideal integrator provides no \DC feedback for the op amp.
To avoid operating the op amp with an open loop, it is necessary to add a finite resistor \R[F] in parallel with \C.
Such \textbf{a practical integrator is the same as the inverting active \ac{lpf}} given in \autoref{sec:inverting_active_lpf_1st}.
Its transfer function given in \eqref{eq:1storder_inverting_active_LPF} simplifies to \eqref{eq:ideal_integrator_TF} as \(\R[F] \to \infty\) since \(s\R[F]C \gg 1 \implies 1 + s\R[F]C \approx s\R[F]C\), so the \R[F] terms in the numerator and denominator cancel.

The practical integrator acts as an inverting amplifier for frequencies below \(\wcutoff = 1/(\R[F]C)\) and as an integrator for frequencies above \wcutoff.
The difference between a \iac{lpf} and integrator is that the frequencies of interest are below \wcutoff for the former and above \wcutoff for the latter. Also, the integrator should minimize \wcutoff and maximize the ``passband'' gain for more ideal behavior;
setting \R[F] as high as possible while staying within the capabilities of the op amp achieves both of these objectives.
Relevant specifications of the op amp for determining \R[F] include its gain-bandwidth product, input offset voltage (which is amplified at the output by the \DC gain of the circuit), and input bias current (which flows through \R[F] and introduces an error voltage).

\subsection{Inverting Differentiator}
\begin{center}
	\begin{circuitikz}
		\draw (0, 0) node[op amp] (opamp) {}
		(opamp.-) to[short,-*] ++(-0.5,0) coordinate(VG)
		(opamp.+) to[short] ++(-0.5,0) (opamp.+ -| VG) to[R,l_=$R\sub{B}$] ++(0,-2) node[ground](G){}
		(opamp.out) to[short,*-o] ++(1,0)
		(VG) to[C,l_=$C$] ++(-2,0) to[R,l_=$R\sub{A}$] ++(-2,0) to[short,-o] ++(-0.5,0) node[above]{\vin}
		(VG) to[short,-*] ++(0,1.25) coordinate(VGU) |- ++(0.5,0) coordinate(RFL) to[R=$R$] ++(2,0) -| (opamp.out) to[short] ++(1,0) node[above]{\vout}
		(VGU) to[short] ++(0,1.5) coordinate(CFL) to[short] (RFL |- CFL) to[C=$C\sub{F}$] ++(2,0) to[short] (CFL -| opamp.out) to[short,-*]  (VGU -| opamp.out);
	\end{circuitikz}
\end{center}

Since differentiation is the opposite of integration, the operating principle of the differentiator circuit is to build an integrator but swap the placement of \R and \C.
The ideal differentiator is also similar to the inverting op amp amplifier except that the resistor in series between the input and the op amp's inverting input is replaced with a capacitor.
The circuit shown above is a practical differentiator which includes some necessary components for a real implementation, but an ideal differentiator has \(\R[A] = \C[F] = 0\) (i.e., the resistor is a short circuit and the feedback capacitor is an open circuit).
For the ideal differentiator, the inverting input's voltage is equal to \gnd (the op amp's input bias current produces a negligible voltage across \R[B]) and the non-inverting input is at a virtual ground.
The capacitor voltage is thus equal to \(\vin(t)\) and by definition the current through it is \(\iin(t)\), so \[\iin(t) = C\odv{\vin(t)}{t}\]

\ac{kcl} on the inverting input node means that \(\iin(t) = -\vout(t)/R\) also, which relates the input to the output:

\[C\odv{\vin(t)}{t} = -\frac{\vout(t)}{R}\]

Solving for \(\vout(t)\), we find that it is proportional to the derivative of \(\vin(t)\):

\begin{equation}
	\vout(t) = -RC\odv{\vin(t)}{t}
	\label{eq:ideal_differentiator_time}
\end{equation}

Another way to represent the transfer function is in the frequency domain.
\ac{kcl} on the inverting input node using the capacitor's impedance \(1/(sC)\) relates the input to the output as \[\frac{\vout(s)}{R} = -\frac{\vin(s)}{1/(sC)}= -\vin(s)sC\]

Rearranging gives the transfer function as

\begin{equation}
	\frac{\vout}{\vin}(s) = -sRC
	\label{eq:ideal_differentiator_TF}
\end{equation}

Unfortunately, the ideal differentiator tends to be unstable and is sensitive to high frequency noise.
Both of these problems are due to the fact that the gain increases as frequency increases -- the circuit's gain is theoretically infinite at very high frequencies (though a real op amp's finite gain and bandwidth will limit the high frequency gain).
To address these problems a practical differentiator introduces \R[A] and \C[F].
Substituting the practical differentiator's impedances into \eqref{eq:inverting_amplifier_TF} we have:

\[\frac{\vout}{\vin}(s) = -\frac{R \parallel 1/(s\C[F])}{\R[A] + 1/(s\C)}\]

From this equation it is clear that at high frequencies (\(s \to \infty\), where \C is a short circuit), \R[A] in series with \C prevents the denominator from approaching 0.
After some algebra we obtain:

\begin{equation}
	\frac{\vout}{\vin}(s) = \frac{-sRC}{(1 + s\R[A] C)(1 + s R \C[F])}
	\label{eq:practical_differentiator_TF}
\end{equation}

This equation shows that the practical differentiator has a zero at \(s = 0\) (like the ideal differentiator) but also poles at \(s = 1/(\R[A] C)\) and \(s = 1/(R \C[F])\), resulting in corner frequencies of \(\f[A] = 1/(2\pi\R[A] C)\) and \(\f[B] = 1/(2\pi R \C[F])\).
The practical differentiator is a second-order \ac{bpf} since its transfer function has the form of \eqref{eq:2ndorderBPF}, though the differentiator is intended to be used for frequencies below the passband.
\f[A] and \f[B] are chosen so that the lower corner frequency is at the maximum signal frequency to be differentiated, and the higher corner frequency is at a low enough frequency to keep the op amp stable.

\section{Second-order low and high pass filters}
% TODO: rename the VCVS filters to Sallen-Key (see ADI filters p. 8.72)
\subsection{Second-order \acs{vcvs} active \acl{lpf}}
\begin{center}
	\includegraphics{schematics/2ndorder_vcvs_LPfilter.PNG}
\end{center}
This second order \ac{vcvs} low pass filter is similar in topology to the the above first order low pass filter except that it includes an extra resistor and capacitor.
The full derivation of the input/output relationship requires a somewhat lengthy manipulation of equations, but it requires only three facts: (1) the voltage at the input terminals, which is $\frac{R_3}{R_3 + R_4}\vout$ (again assuming that $v\sub{MID} = \SI{0}{\V}$), (2) the nodal equation at the non-inverting input, and (3) the nodal equation at the node common to $R_1$, $R_2$, and $C_2$.
This circuit is a second order filter so the transfer function is of the form

\textcolor{red}{
\begin{equation}
\frac{\vout}{\vin}(s) = \Hs = \frac{K b \wcutoff^2}{s^{2} + a\wcutoff s + b\wcutoff^2}
\label{eq:2ndorder_vcvs_LPfilter}
\end{equation}
}

In this case

\begin{equation}
b\wcutoff^2 = \frac{1}{R_1 R_2 C_1 C_2}
\end{equation}
\begin{equation}
a\wcutoff = \frac{1}{C_2}\left(\frac{1}{R_1} + \frac{1}{R_2}\right) - \frac{R_4}{R_2 R_3 C_1}
\end{equation}
\begin{equation}
K = 1 + \frac{R_4}{R_3}
\end{equation}

To minimize the error due to the op amp's input bias currents we need

\begin{equation}
R_3 \parallel R_4 = R_1 + R_2
\end{equation}

With these restrictions and a desired gain $K$ and \wcutoff we have the following equations for deciding the resistor and capacitor values \autocite[118-119]{op-amp-circuits-johnson}:

\begin{equation}
C_1 \leq \frac{(a^2 + 4b(K-1))C_2}{4b}
\end{equation}
\begin{equation}
R_1 = \frac{2}{(a C_2+\sqrt{(a^2 + 4b(K-1))C_2^2 - 4b C_1 C_2})\wcutoff}
\end{equation}
\begin{equation}
R_2 = \frac{1}{bC_1 C_2 R_1\wcutoff^2}
\end{equation}
\begin{equation}
R_3 = \frac{K(R_1 + R_2)}{K-1}, K > 1
\end{equation}
\begin{equation}
R_4 = K(R_1 + R_2)
\end{equation}

The parameters $a$ and $b$ depend on the type of filter desired -- Butterworth or Chebyshev.
Check Butterworth and Chebyshev coefficient tables for these parameter values.
% TODO: add appendix for the Butterworth and Chebyshev a and b coefficient parameters

\subsection{Second-order \acs{vcvs} \acl{hpf}}
\begin{center}
	\includegraphics{schematics/2ndorder_vcvs_HPfilter.PNG}
\end{center}
The \ac{vcvs} second-order high pass filter is the same as the \ac{vcvs} second-order low pass filter but with the resistors replaced with capacitors and the capacitors replaced by resistors (except for the resister divider). Using the  general transfer function for a second-order high pass filter

\textcolor{red}{
\begin{equation}
\frac{\vout}{\vin}(s) = \Hs = \frac{Ks^2}{s^2 + \frac{a}{b}\wcutoff s + \frac{\wcutoff^2}{b}}
\label{eq:2ndorder_vcvs_HPfilter}
\end{equation}
}

the resistors determine the transfer function as follows \autocite[130-131]{op-amp-circuits-johnson}:
\begin{equation}
R_1 = \frac{4b}{(a+\sqrt{a^2 + 8b(K-1)})\wcutoff C}
\end{equation}
\begin{equation}
R_2 = \frac{b}{\wcutoff^2 C^2 R_1}
\end{equation}
\begin{equation}
R_3 = \frac{K R_1}{K-1}, K > 1
\end{equation}
\begin{equation}
R_4 = K R_1
\end{equation}

If no gain is needed (i.e. $K = 1$) $R_3$ can be removed and $R_4$ replaced with a short.

\subsection{Second-order biquad \acl{lpf}}
\begin{center}
	\includegraphics{schematics/2ndorderbiquadLPfilter.PNG}
\end{center}
The biquad filter requires two more op amps than the above \ac{vcvs} filters.
Two of the op amps are used as integrators and the third is an inverter.
The two integrators (which, of course, are low pass filters) form a second order low pass filter.
The transfer function can be determined by nodal analysis and noting the functions of each of the three op amps, but the full analysis is skipped.
Instead, the resistors and capacitors are simply given in terms of the general second-order low pass filter transfer function

\textcolor{red}{
\begin{equation}
\frac{\vout}{\vin}(s) = \Hs = \frac{Kb\wcutoff^2}{s^2 + a\wcutoff s + b\wcutoff^2}
\label{eq:2ndorderbiquadLPfilter}
\end{equation}
}

The resistors relate to the general transfer function as follows:
\begin{equation}
R_4 = \frac{1}{\wcutoff C}
\end{equation}
\begin{equation}
R_1 = \frac{R_4}{Kb}
\end{equation}
\begin{equation}
R_2 = \frac{R_4}{a}
\end{equation}
\begin{equation}
R_3 = \frac{R_4}{b}
\end{equation}

Why use the biquad circuit with its three op amps when a second order filter can be built with only one op amp?
Notice that the above equations for the biquad circuit's resistor and capacitor choices that the biquad is easier to tune than the \ac{vcvs} filters.
In particular, the desired \wcutoff determines the values of $C$ and $R_4$, and with $R_4$ chosen parameter $a$ is determined solely by $R_2$, parameter $b$ can be determined by $R_3$, and with $R_3$ determined $R_1$ can be used to set the filter's gain $K$.
The \ac{vcvs} filters require two different capacitor values and the resistors affect one or more values of $a$, $b$, and $K$ in nontrivial ways.
The biquad circuit's offer of simpler tuning may be worth the two additional op amps. \autocite[120-122]{op-amp-circuits-johnson}

\subsection{Second-order biquad \acl{hpf}}
\begin{center}
	\includegraphics{schematics/2ndorderbiquadHPfilter.PNG}
\end{center}
This circuit is a biquad filter that implements the general second-order high pass filter transfer function

\textcolor{red}{
\begin{equation}
\frac{\vout}{\vin}(s) = \Hs = \frac{Ks^2}{s^2 + \frac{a}{b}\wcutoff s + \frac{\wcutoff^2}{b}}
\label{eq:2ndorderbiquadHPfilter}
\end{equation}
}

with an inverting gain (i.e. $K < 0$).
The resistors relate to the transfer function as follows \autocite[131]{op-amp-circuits-johnson}:
\begin{equation}
R_1 = \frac{b}{a K\wcutoff C}
\end{equation}
\begin{equation}
R_2 = K R_1
\end{equation}
\begin{equation}
R_3 = \frac{b}{\wcutoff C}
\end{equation}
\begin{equation}
R_4 = \frac{1}{K\wcutoff C}
\end{equation}
\begin{equation}
R_5 = \frac{1}{\wcutoff C}
\end{equation}

\section{Band pass filters}

\subsection{\acs{vcvs} band pass filter}
\begin{center}
	\includegraphics{schematics/vcvs_bandpass.PNG}
\end{center}
The general equation for the transfer function of this filter is

\textcolor{red}{
\begin{equation}
\frac{\vout}{\vin}(s) = \Hs = \frac{\alpha \wcutoff s}{s^2+\beta \wcutoff s + \gamma \wcutoff^2}
\label{eq:vcvs_bandpass}
\end{equation}
}

Since this is a band pass filter $\vout/\vin \approx 0$ as $s \to 0$ and $s \to \infty$ but the gain is nonzero in the midband.
The resistors are related to the transfer function as follows \autocite[138-139]{op-amp-circuits-johnson}:
\begin{equation}
R_1 = \frac{2}{\alpha \wcutoff C}
\end{equation}
\begin{equation}
R_2 = \frac{2}{(-\beta+\sqrt{(\alpha - \beta)^2 + 8\gamma})\wcutoff C}
\end{equation}
\begin{equation}
R_3 = \frac{1}{\gamma \wcutoff^2 C^2}\left(\frac{1}{R_1} + \frac{1}{R_2}\right)
\end{equation}
\begin{equation}
R_4 = 2R_3
\end{equation}

\subsection{Biquad band pass filter}
\begin{center}
	\includegraphics{schematics/biquad_bandpass.PNG}
\end{center}
The biquad band pass filter has the same transfer function as the \ac{vcvs} band pass filter:

\textcolor{red}{
\begin{equation}
\frac{\vout}{\vin}(s) = \Hs = \frac{\alpha \wcutoff s}{s^2 + \beta\wcutoff s + \gamma \wcutoff^2}
\label{eq:biquad_bandpass}
\end{equation}
}

with the four resistors related to the transfer function as
\begin{equation}
R_1 = \frac{1}{\alpha \wcutoff C}
\end{equation}
\begin{equation}
R_2 = \frac{1}{\beta \wcutoff C}
\end{equation}
\begin{equation}
R_3 = \frac{1}{\gamma \wcutoff C}
\end{equation}
\begin{equation}
R_4 = \frac{1}{\wcutoff C}
\end{equation}

Again, the \ac{vcvs} band pass filter is simpler than the biquad but the latter is easier to tune than the former. \autocite[140]{op-amp-circuits-johnson}

\section{Band reject filters}

\subsection{Passive Twin-T band reject filter}
\begin{center}
	\includegraphics{schematics/passiveTwinTbandrejectfilter.PNG}
\end{center}
The intuitive way to understand that this circuit is a band reject filter is to realize that these are two T circuits in parallel -- one T circuit is composed of the two $R_1$ resistors and $C_2$ capacitor, and the other T circuit is composed of the two $C_1$ capacitors and $R_2$ resistor.
Consider each T individually:
the T circuit with the two resistors is a low pass filter since $C_2$ shorts the middle node to ground for high frequency signals and passes low frequency signals, and the T circuit with the two capacitors blocks low frequency signals but shorts \vin to \vout for high frequency signals.
Low frequencies can pass through the low pass T circuit and high frequencies can pass through the high pass T circuit, but there is a middle frequency that can pass through neither.
Thus, this Twin-T circuit is a band reject filter.

There are several methods for deriving the notch frequency, but the derivation is lengthy and not presented here.
The result of the derivation \autocite[see][]{cr-bond-twin-t-notch} is

\begin{equation}
	R_1 C_1 = 4R_2C_2
\end{equation}

and

\begin{equation}
\omega^{2} = \frac{1}{2 R_1 R_2 C_2^2}
\end{equation}

Usually the components are chosen such that $R_2 = R_1/2$ and $C_2 = 2C_1$.
In that case, the notch frequency is

\textcolor{red}{
\begin{equation}
f = \frac{\omega}{2\pi} = \frac{1}{2\pi R_1C_1}
\label{eq:passiveTwinTnotchfreq}
\end{equation}
}

The limitation of this circuit is its quality factor $Q = 1/\Delta\Omega$, where $\Delta\Omega$ is the difference between the \SI{-3}{\dB} frequencies just above and below the notch frequency $f$.
For the passive Twin-T band reject filter,

\textcolor{red}{
\begin{equation}
Q = \frac{1}{4}
\label{passiveTwinT_Q}
\end{equation}
}

An active Twin-T band reject filter (which uses the Twin-T topology in the feedback path of an operational amplifier) improves $Q$. \autocite[321]{op-amps-for-everyone}

\subsection{Active Twin-T band reject filter}
\begin{center}
	\includegraphics{schematics/activeTwinTbandrejectfilter.PNG}
\end{center}
The active Twin-T band reject filter adds an operational amplifer to the passive Twin-T filter.
The operational amplifier provides an increased gain and $Q$ since it is an active element.
$R_4$ provides negative feedback and $R_3$ sets the gain in the pass band (the filter circuitry has a gain of appoximately 1 in the pass band, so the circuit is essentially a non-inverting op amp amplifier).
The pass band gain $K$ is thus simply

\textcolor{red}{
\begin{equation}
K = 1 + \frac{R_4}{R_3}
\end{equation}
}

The filter circuitry itself is configured the same as that of the passive Twin-T filter, with the passive version's output node connected to the operational amplifier's non-inverting input and $R_2$ connected to the active filter's output.
Since the filter circuitry is the same, so is the notch frequency $f$.
Assuming the typical case of $R_2 = R_1/2$ and $C_2 = 2C_1$,

\textcolor{red}{
\begin{equation}
f = \frac{1}{2\pi R_1 C_1}
\end{equation}
}

It can be shown \autocite[322]{op-amps-for-everyone} that the active Twin-T filter's $Q$ is

\textcolor{red}{
\begin{equation}
Q = \frac{1}{2(2-k)} = \frac{R_3}{2(R_3 - R_4)}
\end{equation}
}

With $K$ and $Q$ given in terms of the circuit components, the overall transfer function can be written in terms of the circuit components as well:

\textcolor{red}{
\begin{equation}
\frac{\vout}{\vin}(s) = \Hs = \frac{K(s^2 + 1)}{s^2 + \frac{s}{Q} + 1} = \frac{\frac{R_3 + R_4}{R_3}(s^2 + 1)}{s^2 + \frac{2(R_3 - R_4)}{R_3}s+1}
\end{equation}
}

\subsection{Passive Wien-Robinson band reject filter}
\begin{center}
	\includegraphics{schematics/passiveWien-Robinsonbandrejectfilter.PNG}
\end{center}
The Wien-Robinson band reject filter takes a single-ended input and produces a differential output.
It can be analyzed intuitively by realizing that it is composed of a voltage divider ($R_3$ and $R_4$) in parallel with a band pass filter.
$R_1$, $R_2$, $C_1$, and $C_2$ form a band pass filter because $C_1$ blocks low frequencies (no current passes through $C_1$ to create a voltage across $R_2$ and $C_2$) and $C_2$ shorts the band pass filter's output (the negative terminal of \vout) to ground, but midrange frequencies are passed to the output.
The overall circuit acts as a band reject filter because at low and high frequencies the negative terminal of \vout is grounded so that \[\vout = \frac{R_3}{R_2 + R_3}\vin\]
At middle frequencies \vin is passed to the negative terminal of \vout so that the terminals of \vout are equal and $\vout = 0$.

This circuit suffers from a low $Q$ \autocite[323]{op-amps-for-everyone}, just as the passive Twin-T band reject filter (the two have similar values of $Q$).
It too can be improved with the use of operational amplifiers.

Typically the component values are chosen such that $R_1 = R_2$ and $C_1 = C_2$.
In that case the notch frequency \autocite[324]{op-amps-for-everyone} is

\textcolor{red}{
\begin{equation}
f = \frac{1}{2\pi R_1 C_1}
\label{eq:passiveWienRobinson}
\end{equation}
}

\subsection{Active Wien-Robinson band reject filter}
\begin{center}
	\includegraphics{schematics/activeWien-Robinsonbandrejectfilter.PNG}
\end{center}
The active Wien-Robinson band reject filter uses two operational amplifiers to improve the passive Wien-Robinson filter's $Q$ and gain.
The resistors and capacitors which compose the passive Wien-Robinson filter are labeled the same in the above schematic of the active version as in the passive version from before.
The inputs of the operational amplifier which drives the overall output are driven by the filter's differential output.
The other operational amplifier is configured simply as an inverting amplifier.
The transfer function \autocite[323]{op-amps-for-everyone} can be written as

\textcolor{red}{
\begin{equation}
\frac{\vout}{\vin}(s) = \Hs = \frac{\frac{R_6 R_7}{R_5(R_6 + R_7)}(s^2 + 1)}{s^2 +\frac{3R_7}{R_6 + R_7}s+1}
\label{eq:activeWienRobinsonbandreject}
\end{equation}
}

If $R_1 = R_2$ and $C_1 = C_2$ as is typical, then

\textcolor{red}{
\begin{equation}
f = \frac{1}{2\pi R_1 C_1}
\end{equation}
}

since the filtering circuitry is unchanged from the passive version.
$Q$ can be determined by inspection since it is the coefficient of the first order $s$ term,

\textcolor{red}{
\begin{equation}
Q = \frac{3R_7}{R_6 + R_7}
\end{equation}
}

and the active filter's passband gain $K$ can also be determined by inspection:

\textcolor{red}{
\begin{equation}
K = \frac{R_6 R_7}{R_5(R_6 + R_7)}
\end{equation}
}

The active Wien-Robinson band reject filter differs from its active Twin-T counterpart in that the passband gain $k$ can be chosen without affecting the quality factor $Q$.

\subsection{\ac{vcvs} band reject filter}
\begin{center}
	\includegraphics{schematics/vcvs_bandreject.PNG}
\end{center}
This band reject filter's transfer function can be written in the form

\textcolor{red}{
\begin{equation}
\frac{\vout}{\vin}(s) = \frac{s^2 + \wcutoff^2}{s^2 + \frac{\wcutoff}{Q}s + \wcutoff^2}
\label{eq:vcvs_bandreject}
\end{equation}
}

The op amp is configured as a voltage follower and, since it is the only active device in the circuit, the overall gain of this filter can never exceed unity even in the pass band.
To achieve the desired transfer function the resistors must be chosen as follows \autocite[145-146]{op-amp-circuits-johnson}:
\begin{equation}
R_1 = \frac{1}{2Q\wcutoff C}
\end{equation}
\begin{equation}
R_2 = \frac{2Q}{\wcutoff C}
\end{equation}
\begin{equation}
R_3 = \frac{R_1 R_2}{R_1 + R_2}
\end{equation}

\subsection{Biquad band reject filter}
\begin{center}
	\includegraphics{schematics/biquad_bandreject.PNG}
\end{center}
A band reject filter's transfer function can also be written in the form 

\textcolor{red}{
\begin{equation}
\frac{\vout}{\vin}(s) = \Hs = \frac{\alpha (s^2 + \wcutoff^2)}{s^2 + \beta \wcutoff s + \gamma \wcutoff^2}
\label{eq:biquad_bandreject}
\end{equation}
}

Unlike the \ac{vcvs} band reject filter above, this biquad filter can provide an inverting gain greater than unity (the $\alpha$ term), and it can also achieve a much higher $Q$.
The resistors relate to the transfer function as \autocite[146-148]{op-amp-circuits-johnson}
\begin{equation}
R_1 = \frac{1}{\alpha \beta \wcutoff C}
\end{equation}
\begin{equation}
R_2 = \alpha R_1
\end{equation}
\begin{equation}
R_3 = \frac{1}{\sqrt{\gamma} \wcutoff C}
\end{equation}
\begin{equation}
R_4 = \frac{1}{\alpha \wcutoff C}
\end{equation}
\begin{equation}
R_5 = \frac{\sqrt{\gamma}}{\alpha \wcutoff C}
\end{equation}
\begin{equation}
R_6 = \frac{1}{\wcutoff C}
\end{equation}

\section{All pass filters}
\subsection{First-order all pass filter}
\begin{center}
	\includegraphics{schematics/1storderallpassfilter.PNG}
\end{center}
% TODO: see https://en.wikipedia.org/wiki/All-pass_filter, which includes a version that swaps the R and C at the non-inverting input and passive versions
This circuit \autocite[328]{op-amps-for-everyone} has a gain of 1 at low frequencies and -1 at high frequencies.
At low frequencies, neither the capacitor nor the non-inverting input of the operational amplifier draw any current, so there is no voltage drop across $R_2$ and the non-inverting input has a voltage equal to \vin.
The inverting input also has a voltage equal to \vin since the operational amplifier's inputs must be at (approximately) equal voltage, so there is no voltage drop across the $R_1$ connected to \vin.
Since there is no voltage drop across (or current through) the $R_1$ connected to \vin, there is no current through (or voltage drop across) the $R_1$ connected to \vout. Thus, $\vout = \vin$ at low frequencies.
At high frequencies, the non-inverting input of the operational amplifier is shorted to ground since the capacitor has a very low impedance.
$R_2$ does nothing to the circuit since it is only connected to \vin and \gnd and the circuit looks like an inverting op amp amplifier, which of course has a gain of -1 when the gain and feedback resistors ($R_1$, in this case) are equal.

It is not intuitively obvious, however, that the circuit has a gain of magnitude 1 in the middle of the frequency spectrum, so we need to derive the transfer function.
Let the voltage at the operational amplifier's inputs be $v_x$.
$R_2$ and $C_1$ act as a voltage divider for \vin at the operational amplifier's inverting input, so

\begin{equation}
	v_x = \frac{\frac{1}{sC_1}}{R_2 + \frac{1}{sC_1}}\vin = \frac{\vin}{1+s R_2 C_1}
\end{equation}

The operational amplifier's non-inverting input is also at voltage $v_x$ (ideally), so use \ac{kcl} at the non-inverting input node:

\begin{equation}
\frac{\vout - v_x}{R_1} = \frac{v_x - \vin}{R_1}
\end{equation}

Substituting for $v_x$, we have

\begin{equation}
\frac{\vout}{R_1}-\frac{\vin}{R_1(1+s R_2 C_1)} = \frac{\vin}{R_1(1+s R_2 C_1)}-\frac{\vin}{R_1}
\end{equation}

Rearranging, we have

\begin{equation}
\vout = \vin\left(\frac{2}{1+s R_2 C_1}-1\right)
\end{equation}

Rearranging further, the transfer function is thus

\textcolor{red}{
\begin{equation}
\frac{\vout}{\vin}(s) = \frac{1 - s R_2 C_1}{1 + s R_2 C_1}
\label{eq:1storderallpassfilter}
\end{equation}
}

The transfer function reveals a zero at $s = 1/(R_2 C_1)$ and a pole at $s = -1/(R_2 C_1)$ (technically, the operational amplifier introduces its own poles to the system so that the gain does actually go to zero at very high frequencies, but we are assuming that the frequencies of operation are well below the operational amplifier's poles).
In any case, the gain has constant magnitude $\abs{H(s)} = \abs{H(\jmath\omega)}$ across the (operational) frequency spectrum since

\begin{equation}
\abs{H(\jmath\omega)}= \frac{\sqrt{1+(-\omega R_2 C_1)^2}}{\sqrt{1+(\omega R_2 C_1)^2}} = 1
\end{equation}

so the circuit is an all pass filter.
The point of an all pass filter is that it can change a system's phase response across the frequency spectrum -- the phase changes from $0$ to $-\pi$ radians from low to high frequency.
More specifically, the phase $\angle H(s) = \angle H(\jmath\omega)$ is

\textcolor{red}{
\begin{equation}
\angle H(\jmath\omega) = \tan^{-1}(-\omega R_2 C_1) - \tan^{-1}(\omega R_2 C_1) = -2\tan^{-1}(\omega R_2 C_1)
\label{eq:1storderallpassfilter_angle}
\end{equation}
}

Although the values of $R_2$ and $C_1$ do not affect the magnitude of the circuit's gain across the frequency spectrum, they must be chosen appropriately to shape the circuit's phase response over frequency as desired.
$R_1$ can be any reasonable value, of course, since the $R_1$ resistors do not directly affect the transfer function.

\subsection{Second-order all pass filter}
\begin{center}
	\includegraphics{schematics/2ndorder_allpass.PNG}
\end{center}
A second order all pass filter has the transfer function

\textcolor{red}{
\begin{equation}
\frac{\vout}{\vin}(s) = H(s) = \frac{K(s^2 - a\wcutoff s + b\wcutoff^2)}{s^2 + a\wcutoff s + b\wcutoff^2}
\label{eq:2ndorder_allpass}
\end{equation}
}

since this transfer function has two poles and

\begin{equation}
\abs{H(\jmath\omega)} = K\frac{\sqrt{(b\wcutoff^2-\omega^2) + (-a\wcutoff\omega)^2}}{\sqrt{(b\wcutoff^2-\omega^2) + (a\wcutoff\omega)^2}} = K
\end{equation}

The phase is

\textcolor{red}{
\begin{equation}
\angle H(\jmath\omega) = \tan^{-1}\left(-\frac{b\wcutoff^2-\omega^2}{a\wcutoff\omega}\right) - \tan^{-1}\left(\frac{b\wcutoff^2-\omega^2}{a\wcutoff\omega}\right) = -2\tan^{-1}\left(\frac{b\wcutoff^2-\omega^2}{a\wcutoff\omega}\right)
\label{eq:2ndorder_allpass_angle}
\end{equation}
}

This circuit implements the second order all pass filter transfer function, with the resistors chosen such that
\begin{equation}
a\omega_{o} = \frac{2}{R_2 C}
\end{equation}
\begin{equation}
b\omega_{o}^2 = \frac{1}{R_1 R_2 C^2}
\end{equation}
\begin{equation}
K = \frac{R_4}{R_3 + R_4}
\end{equation}
\begin{equation}
4R_1 R_4 = R_2 R_3
\end{equation}

For minimum \DC offset, choose \autocite[151-153]{op-amp-circuits-johnson}:

\begin{equation}
R_2 = \frac{R_3 R_4}{R_3 + R_4}
\end{equation}